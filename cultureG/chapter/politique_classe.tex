\chapter{Politique}
\section{Libéralisme politique / économique}

\paragraph{Libéralisme politique} Né au 19\ieme siècle, le libéralisme économique est le mouvement majeur à cette période. Ces idées s'inspirent de la révolution française, le libéralisme politique stipule que tous les individus vivant sur un même territoire sont égaux. Cela s'oppose aux sociétés d'avant (absolutisme, monarchie absolue, ...) où des individus différents ayant des droits différents en fonction de leur classe. Maintenant, des personnes habitant sur le même territoire partage les mêmes droits.

Libéralisme : ce qui prime, c'est l'individu. L'\'Etat doit permettre à un individu de se développer. Les libéraux revendiquent pour la liberté (s'oppose au mouvement national qui combat pour l'unification / l'indépendance d'un peuple).

\paragraph{Libéralisme économique} Prône la liberté pour les entreprises. S'opposent à ce que l'\'Etat impose des règles contre les entreprises. Plus les entrepreneurs sont libres, plus ils peuvent faire prospérer l'économie : ils doivent embaucher des personnes ce qui crée de l'emploi. Les entrepreneurs veulent une liberté totale : liberté sur les salaires, sur les conditions de travail ainsi que sur le temps de travail. Les lois sociales (comme celles limitant le travail des enfants, sur le temps de travail) sont un frein à l'économie.




\section{Socialisme}
Ce mouvement condamne les inégalités sociales. Ils souhaitent une société plus juste, basée sur la redistribution des richesses. Certains sont réformistes, d'autres révolutionnaires.Les socialistes veulent améliorer les conditions de vie sans toutefois imposer un changement par la violence, par opposition aux marxistes. 

\section{Marxisme}
Abolition des classes. Société capitaliste : \'Etat = dictature de la bourgeoisie. Une classe : \textbf{les bourgeois}  possèdent des terres/entreprises et exploitent l'autre classe qui ne possèdent rien et qui pour subsister doivent travailler : \textbf{les prolétaires}. Les bourgeois veulent du profit, de la \textbf{plus-value}. Pour Marx et Engels, il faut abolir les classes pour ne plus avoir une classe qui exploite l'autre. Cela passe par la \textit{dictature du prolétariat}. Les prolétaires doivent s'emparer des entreprises par la force car le capitalisme ne va pas céder au communisme sans résister, les bourgeois voudront récupérer leurs avantages à tout prix. L'\'Etat qui favorisant la bourgeoisie sur le prolétariat s'effacera petit à petit pour enfin disparaître : c'est le communisme, tous les moyens de productions sont mis en commun.

Prolétariat doit s'emparer du pouvoir / moyen production pour créer soc égalitaire / sans classe
\vspace{0.3cm}

\textbf{$\underbrace{\text{Dictature de la bourgeoisie}}_{\text{société actuelle}}  \rightarrow \underbrace{\text{dictature du prolétariat}}_{\text{ moyens production au prolétaires qui dominent}} \rightarrow \underbrace{\text{communisme}}_{\text{société sans classe}}.$}
\section{Catholicisme sociaux}
Le pape Léon 13 s'est exprimé au sujet de l'économie : il souhaite un système où l'on puisse protéger les femmes et les enfants, assurer un juste salaire pour les ouvriers sans toutefois pousser lutte des classes car celle-ci entraîne de la haine/jalousie

